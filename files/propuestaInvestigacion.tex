% Paquetes describiendo y usados en el documento %
\documentclass[12pt,a4paper]{article}
\usepackage[utf8]{inputenc}
\usepackage{fullpage}
\usepackage[T1]{fontenc}
\usepackage{parskip}
\usepackage[version=4]{mhchem}
\usepackage{chemfig}
\usepackage{pstricks}
\usepackage{plain}
\usepackage{float}
\usepackage{multicol}
\usepackage{changepage}
\usepackage{amsfonts}
\usepackage{amsmath}
\usepackage{amssymb}
\usepackage{graphicx}
\usepackage{tikz}
\usepackage{scalerel}
\usepackage{pict2e}
\usepackage{tkz-euclide}
\usetikzlibrary{calc, arrows, decorations}
\definecolor{applegreen}{rgb}{0.55, 0.71, 0.0}
\usepackage{xcolor}
\usepackage{ragged2e}
\usepackage{pgfplots}
\pgfplotsset{compat=1.18}
\usepackage{babel}
\setlength{\parskip}{-0.03cm}

\pgfdeclaredecoration{ddbond}{initial}{
  \state{initial}[width=4pt]{
    \pgfpathlineto{\pgfpoint{4pt}{0pt}}
    \pgfpathmoveto{\pgfpoint{2pt}{2pt}}
    \pgfpathlineto{\pgfpoint{4pt}{2pt}}
    \pgfpathmoveto{\pgfpoint{4pt}{0pt}}
  }
  \state{final}{
    \pgfpathlineto{\pgfpointdecoratedpathlast}
  }
}
\tikzset{lddbond/.style={decorate, decoration=ddbond}}
\tikzset{rddbond/.style={decorate, decoration={ddbond, mirror}}}

\newcommand*{\doi}[1]{\href{http://dx.doi.org/#1}{#1}}
\usepackage{hyperref}
\hypersetup{colorlinks=true,linkcolor=blue,urlcolor=blue,filecolor=blue, citecolor=blue}

\renewcommand{\baselinestretch}{1.5}
\renewcommand\refname{Referencias}
\renewcommand\contentsname{Contenido}
\renewcommand\tablename{Tabla}
\DeclareUnicodeCharacter{2009}{FIXME}

%%%%%%%%%%%%%%%%%%%%%%%%%%%%%%%%%%%%%%%%%%%%%%%%%%%%%%%%%%%%%%%%%%%%%%%%%%%%%%%%%%%%%%%%%%%

% Márgenes del documento %
\usepackage{vmargin}
\setpapersize{A4}
\setmargins{2.5cm}        % margen izquierdo
{1 cm}                    % margen superior
{16.5cm}                  % anchura del texto
{23.4cm}                  % altura del texto
{11pt}                    % altura de los encabezados
{1.5cm}                     % espacio entre el texto y los encabezados
{0pt}                     % altura del pie de página
{2cm}           
%%%%%%%%%%%%%%%%%%%%%%%%%%%%%%%%%%%%%%%%%%%%%%%%%%%%%%%%%%%%%%%%%%%%%%%%%%%%%%%%%%%%%%%%%%%

% Omitir %
\title{´}
\author{Heidy...}
\date{}
%%%%%%%%%%%%%%%%%%%%%%%%%%%%%%%%%%%%%%%%%%%%%%%%%%%%%%%%%%%%%%%%%%%%%%%%%%%%%%%%%%%%%%%%%%%

% Inicio del documento %
\begin{document}

%\begin{titlepage}

  \begin{center}

      %\vspace{0.1cm}
	  \LARGE \textbf{C\'alculo de los espinores de campo ligando del \ce{ReF6} a partir de medidas experimentales}
      
      \vspace{1.3cm}
      \Large \textbf{Autor:}
      \vspace{0.5cm}
      
      {\Large Heidy...}
      
      \vspace{1.3cm}
      
      \Large \textbf{Director:}
      \vspace{0.5cm}
      
      \Large Prof. Jhon Fredy Pérez Torres
      
      \vspace{1.3cm}
      
      \begin{figure}[h]
          \centering
          \includegraphics[scale=0.6]{UIS.png}
      \end{figure}
      
      {\large Universidad Industrial de Santander}\\
      \vspace{0.1cm}
      {\large Facultad de Ciencias}\\
      \vspace{0.1cm}
      {\large Escuela de Qu\'imica}\\
      \vspace{0.1cm}
      {\large Grupo de Bioqu\'imica Te\'orica y Computacional}\\
      \vspace{0.1cm}
      {\large Bucaramanga}\\
      \vspace{0.1cm}
      {\large Agosto--2024}
  \end{center}
%\end{titlepage}

\newpage

\tableofcontents
\newpage

\section{Introducci\'on}
Los complejos de metales de transición son compuestos en los que un ion metálico central se encuentra rodeado
y enlazado a uno o más ligandos, que son átomos, iones o moléculas donadores de pares de electrones. As\'i, el
metal de transici\'on es un \'acido de Lewis y los ligandos bases de Lewis.
La naturaleza de estos enlaces y la disposición geométrica de los ligandos
alrededor del ion metálico influyen en las propiedades químicas y físicas de estos compuestos.
La aplicaci\'on de la ecuación de Schrödinger en conjunto con la teoría de perturbaciones a estos compuestos
da origen a una teor\'ia que se conoce con dos nombres, la Teor\'ia del Campo Cristalino (TCC) que se aplica a
compuestos i\'onicos no conductores, y la Teor\'ia del Campo Ligando (TCL) que se aplica a compuestos moleculares.
En ambos casos el objetivo es estudiar las propiedades de los electrones en orbitales d del metal de transici\'on.
La diferencia entre la TCC y TCL es sutil \cite{Smith2011} y por tanto en este documento nos referiremos todo el
tiempo a la TCL indistintamente a si se aplica a s\'olidos ionicos o compuestos moleculares.
La TCL proporciona información valiosa sobre la naturaleza del enlace metal-ligando y sobre las propiedades \'opticas
y magnéticas de los complejos.

\section{Plantemiento del problema}
La TCL se basa en la mecánica cuántica no relativista, de all\'i que sus resultados sean exitosos en la interpretación
de las propiedades de complejos que involucran metales de transición de la primera serie, los elementos 3d.
Naturalmente cuando se aplica a complejos de metales de transición de la segunda serie (4d), y en especial
de la tercera serie (5d), se encuentran desviaciones significativas.
Obviamente una de las principales limitaciones de la TCL es su incapacidad para explicar de manera precisa los efectos
que se derivan del acoplamiento espín-órbita, que es un fenómeno de gran importancia en átomos pesados.
Estos efectos influyen en la estructura electrónica de los átomos y, por consiguiente, en las propiedades de los
compuestos que forman.
Un ejemplo claro del fracaso de la TCL estandar en metales 5d es el hexafluoruro de renio (\ce{ReF6}), un complejo molecular que presenta dos
transiciones d-d en lugar de una como lo predice la teor\'ia para complejos octa\'edricos ${\rm d}^1$.
Para tratar de remediar el problema, algunos autores optaron por incorporar la interacci\'on-esp\'in orbita en la TCL,
logrando explicar el por qu\'e se observan dos transiciones electronicas d-d, pero no pudieron explicar de manera satisfactoria
el momento dipolar magn\'etico efectivo del \ce{ReF6} \cite{Moffitt1959,Selig1962}. Aunque el uso de funciones de onda relativista en
la teor\'ia del campo ligando se remonta a 1960 \cite{Moffitt1959,Liehr1960,Wybourne1965,Basu1982},
no se encuentra en la literatura una teor\'ia estrictamente relativista, tal vez por la falta de datos experimentales de la \'epoca
que insentivara el desarrollo de la teor\'ia, o la falta de poder computacional, no lo sabemos.
El problema central de esta investigación es la ausencia de una teor\'ia actualizada que explique de manera
consistente y satisfactoria las propiedades ópticas y magnéticas de complejos de metales de transición de la
tercera serie, donde los efectos relativistas son fundamentales \cite{Neese1998}.
El grupo de investigaci\'on en Bioqu\'imica Te\'orica y Computacional ha iniciado el desarrollo de un marco te\'orico que en principio
permite describir satisfactoriamente las propiedades \'opticas y magn\'eticas de \'atomos 5d$^1$ en campos octa\'edricos \cite{jperez2024a}.
La teor\'ia relativista del campo de los ligandos (TRCL) parte de la ecuación de Dirac \cite{Dirac1928}, y a diferencia de la ecuaci\'on
de Schr\"odinger que emplea orbitales, emplea espinores de Dirac.
Este hecho es una de las principales diferencias entre ambas teor\'ias. Por ejemplo, para complejos octa\'edricos, en la TCL estandar
los orbitales de campo ligando, conocidos como orbitales $t_{2g}$ y $e_g$, no dependen del par\'ametro de campo $Dq$. Son combinaciones
lineales fijas de los orbitales d. La TRCL predice que los espinores de campo ligando $\gamma_8$ y $\gamma_7$ dependen no solo del
par\'ametro de campo $Dq$, si no que tambi\'en depende de la constante de acomplamiento esp\'in-\'orbita $\xi_{n{\rm d}}$ y de la
raz\'on relativista $p/q$ \cite{jperez2024a}.
En esta tesis se pretende calcular los espinores de campo ligando del hexafluoruro de renio a partir de medidas experimentales,
espec\'ificamente a partir de las energ\'ias de las transiciones d-d observadas en el espectro de absorci\'on \cite{Rotger1999}
y de la susceptibilidad magn\'etica \cite{Selig1962}.

\section{Objetivos}

\subsection{Objetivo general}

\subsection{Objetivos espec\'ificos}

\section{Marco te\'orico}

\section{Metodolog\'ia}

\section{Resultados esperados}

\section{Cronograma de actividades}

\section{Presupuesto}

\newpage
\bibliographystyle{unsrturl}
\bibliography{ref.bib}

\end{document}
